\documentclass[pdftex,twocolumn,10pt]{article}
\usepackage{amsfonts,amssymb,amsmath}
\usepackage{graphicx}
\usepackage{url}
\usepackage{cite}
\usepackage{booktabs}
\usepackage{titlesec}

\title{Simulating the Stability-Proportionality Trade-off: \\ A Computational Counterfactual Analysis of Proportional Representation in the Indian Parliamentary System}
\author{Ayush Maurya \\ \textit{Your Institution}}
\date{\today}

\begin{document}

\maketitle

\begin{abstract}
The ``Stability Anxiety'' has long dominated the debate on electoral reform... By simulating the 2024 general elections across three societal personas, we quantify the trade-off between representativeness (Gallagher 0.0663) and durability (MTTF 4.86). Our findings demonstrate that while PR offers superior fairness (Gallagher 0.06 vs FPTP 0.56), the ``Polarization Trap'' reduces stability by 70\% in polarized contexts. 
\end{abstract}

\section{Introduction}
India's adoption of the FPTP system resulted in a disproportionality gap of 0.5616.

\section{Theoretical Framework}
Giovanni Sartori's typology of party systems suggests that ideological distance is the primary driver of instability. In India, ``Stability Anxiety'' is often used to justify FPTP, despite reports from the Law Commission supporting PR shifts \cite{LawCommission1999, LawCommission2015}.

\section{Methodology}
We modeled 100,000 agents. Convergence was verified at the 1:10,000 resolution level.

\section{Simulation Results}
Our results identify a ``Polarization Trap'' where MTTF drops to 1.31 years in polarized contexts, while regional systems remain stable at 4.05 years. Centrist PR (5\%) achieves a Gallagher Index of 0.0663.

\section{Conclusion}
Institutional design can mitigate instability, but societal cohesion remains the primary driver of durability.

\section*{Acknowledgements}
The author thanks the BharatSim framework contributors and Ashoka University for their foundational ABM research.

\bibliographystyle{IEEEtran}
\bibliography{references}

\end{document}
